\documentclass[12pt,a4paper]{pptec}
\setmainfont[Ligatures=TeX,AutoFakeBold=0.5,AutoFakeSlant=0.25]{Khmer OS}% roman font
\setsansfont[Ligatures=TeX,AutoFakeBold=0,AutoFakeSlant=0.25]{Khmer OS Muol Light}% sans serif font
\setmonofont[Ligatures=TeX,AutoFakeBold=0.5,AutoFakeSlant=0.25]{Khmer OS Bokor}% typewriter font
\begin{document}
	\chapter{ទំហំសមាមាត្រ និងភាគរយ}
    \begin{objective}\strut
        \begin{itemize}
            \item កំណត់ទំហំពីរសមាមាត្រគ្នា
            \item ដោះស្រាយចំណោទដែលទាក់ទងនឹងសមាមាត្រ
            \item ដោះស្រាយចំណោទដែលទាក់ទងនឹងភាគរយ និងកិច្ចការជំនួញ។
        \end{itemize}
    \end{objective}
    \section{ទំហំសមាមាត្រ}
    \subsection{ទំហំសមាមាត្រ}
    \begin{generality}
        $ y $ និង $ x $ ជាទំហំសមាមាត្រស្របនឹង $ a\neq 0 $ ជាមេគុណសមាមាត្រ\\
        គេបាន $ \frac{y}{x}=a $ ឬ $ y=ax $~។
    \end{generality}
    \subsection{លក្ខណៈនៃសមាមាត្រស្រប}
    \subsubsection{លក្ខណៈទី១}
    \begin{tcolorbox}[title={លក្ខណៈទី១}]
        ក្នុងសមាមាត្រ ផលគុណតួចុងស្មើនឹងផលគុណតួមធ្យម\\
        $ \frac{y_1}{x_1}=\frac{y_2}{x_2} $ សមមាត្រ $ x_2y_1=x_1y_2 $~។
    \end{tcolorbox}
    \subsubsection{លក្ខណៈទី២}
    \begin{tcolorbox}[title={លក្ខណៈទី២}]\strut
        \begin{itemize}
            \item $ x_1 $ និង $ x_2 $ ជាពីរចំនួនខុសពីសូន្យដែល $ x_1+x_2\neq 0 $~។ បើ $ \frac{y_1}{x_1}=\frac{y_2}{x_2} $ នោះ $ \frac{y_1}{x_1}=\frac{y_2}{x_2}=\frac{y_1+y_2}{x_1+x_2} $~។
            \item ដូចគ្នាដែរ $ x_1,\; x_2,\; x_3 $ ជាបីចំនួនខុសពីសូន្យដែល $ x_1+x_2+x_3\neq 0 $\\
            បើ $ \frac{y_1}{x_1}=\frac{y_2}{x_2}=\frac{y_3}{x_3} $ នោះ $ \frac{y_1+y_2+y_3}{x_1+x_2+x_3} $~។
        \end{itemize}
    \end{tcolorbox}
    \subsection{ទំហំសមាមាត្រច្រាស}
    \begin{generality}
        បើ $ y $ និង $ x $ ជាទំហំសមាមាត្រច្រាសនិង $ a\neq 0 $ ជាមេគុណសមាមាត្រច្រាស នោះ $ xy=a $ ឬ $ y=\frac{a}{x} $~។
    \end{generality}
    \section{ភាគរយ}
    \subsection{ការសរសេរបរិមាណមួយជាភាគរយនៃបរិមាណមួយទៀត}
    \begin{generality}
        ដើម្បីសរសេរបរិមាណ $ a $ មួយជាភាគរយនៃបរិមាណ $ b $ មួយទៀតគេត្រូវ
        \begin{itemize}
            \item សរសេរ $ a $ ជាប្រភាគនៃ $ b $ មានន័យថា $ \frac{a}{b} $
            \item គុណប្រភាគ $ \frac{a}{b} $ និង $ 100\% $~។
        \end{itemize}
    \end{generality}
    \subsection{ភាគរយប្រាក់ចំណេញនិងប្រាក់ខាត}
    \begin{tcolorbox}
        \begin{align*}
        \textnormal{ភាគរយប្រាក់ចំណេញ}&=\frac{\textnormal{ប្រាក់ចំណេញ}}{\textnormal{ប្រាក់ថ្លៃដើម}}\times 100\%\\
        \textnormal{ភាគរយប្រាក់ខាត}&=\frac{\textnormal{ប្រាក់ខាត}}{\textnormal{ប្រាក់ថ្លៃដើម}}\times 100\%
        \end{align*}
    \end{tcolorbox}
    \subsection{ការបញ្ចុះតម្លៃ}
    \begin{tcolorbox}
        \begin{equation*}
        \textnormal{ភាគរយការបញ្ចុះតម្លៃ}=\frac{\textnormal{ការបញ្ចុះតម្លៃ}}{\textnormal{ថ្លៃលក់ខាងដើម}}\times 100\%
        \end{equation*}
    \end{tcolorbox}
    \subsection{បម្រែបម្រួលភាគរយ}
    \subsection{ការប្រាក់សាមញ្ញ}
    \begin{tcolorbox}
        \begin{itemize}
            \item \texttt{ការប្រាក់}៖ ជាប្រាក់ចំណេញបានមកពីប្រាក់ឲ្យគេខ្ចីក្នុងរយៈពេលមានកំណត់តាងដោយ $ I $~។
            \item \texttt{ប្រាក់ដើម}៖ ជាប្រាក់ខ្ចីគេ ឬឲ្យគេខ្ចីតាងដោយ $ P $~។
            \item \texttt{អត្រា}៖ ជាការប្រាក់ក្នុងមួយឆ្នាំលើប្រាក់ដើម $ 100 $ រៀល តាងដោយ $ R $~។
            \item \texttt{ប្រាក់សរុប}៖ ជាការប្រាក់និងប្រាក់ដើមរួមគ្នា។
            \item \texttt{រយៈពេល}៖ ជាអំឡុងពេលដែលបានខ្ចីប្រាក់គេ ឬឲ្យគេខ្ចីតាងដោយ $ T $~។
        \end{itemize}
    \end{tcolorbox}
    \begin{tcolorbox}[title={ការប្រាក់សាមញ្ញ}]
        បើ $ P $ ជាប្រាក់ដើមដែលបានខ្ចីគេ ឬឲ្យគេខ្ចី ដោយគិតការប្រាក់តាមអត្រា $ R\% $ ក្នុងមួយឆ្មាំសម្រាប់រយៈពេល $ T $ ឆ្នាំ នោះការប្រាក់គឺ $ \tcbhighmath{I=\frac{PRT}{100}} $~។
    \end{tcolorbox}
    \subsection{ការប្រាក់សមាស}
    \begin{tcolorbox}[title={ការប្រាក់សមាស}]
        រាល់ដំណាច់ឆ្នាំ គេបន្ថែមលើប្រាក់ដើមនូវការប្រាក់កន្លងទៅ គេក៏បានប្រាក់ដើមថ្មី ហើយឆ្នាំក្រោយគេគិតការប្រាក់លើប្រាក់ដើមថ្មីដោយអត្រាដដែល។
    \end{tcolorbox}
    \subsection{ការទិញបណ្ដាក់}
    \section{លំហាត់}
    \begin{enumerate}
        \item ចូររកតម្លៃ $ x $ ក្នុងករណីខាងក្រោម
        \begin{enumerate}
            \item $ 4:7=x:5 $
            \item $ x:8=99:5 $
            \item $ 1\, km:32\, m=250\,g :x\, g $
        \end{enumerate}
        \item
        \begin{enumerate}
            \item គេឱ្យ $ a:b=5:18 $ និង $ a+b=138 $ ចូររកតម្លៃនៃ $ b $~។
            \item គេឱ្យ $ x:y=3:5 $ និង $ x+y=200 $ ចូររកតម្លៃនៃ $ x $~។
        \end{enumerate}
        \item បើ $ a,b,c $ និង $ d $ ជាចំនួនវិជ្ជមាន។ បង្ហាញលក្ខណៈនៃសមាមាត្រ $ \frac{a}{b}=\frac{c}{d} $ ដូចខាងក្រោម
        \begin{enumerate}
            \item $ ad=bc $
            \item $ \frac{a}{c}=\frac{b}{d} $
            \item $ \frac{d}{b}=\frac{c}{a} $
            \item $ \frac{d}{c}=\frac{b}{a} $
            \item $ \frac{a+b}{b}=\frac{c+d}{d} $
            \item $ \frac{a-b}{b}=\frac{c-d}{d} $
            \item*(2) បើ $ c\neq d $ នោះ $ \frac{a-b}{c-d}=\frac{a}{c} $
            \item*(4) បើ $ a\neq b $ និង $ c\neq d $ នោះ $ \frac{a+b}{a-b}=\frac{c+d}{c-d} $~។
        \end{enumerate}
        \item គេឱ្យ $ a:b:c=6:7:9 $
        \begin{enumerate}
            \item រកតម្លៃនៃ $ a $ និង $ c $ បើ $ b=21\, cm $
            \item រកតម្លៃនៃ $ a $ និង $ b $ បើ $ c=720\, g $~។
        \end{enumerate}
        \item 
        \begin{enumerate}
            \item គេឱ្យ $ a:b:c=7:13:20 $ និង $ a+b+c=520\,\textnormal{៛} $~។ ចូររកតម្លៃនៃ $ a,b $ និង $ c $~។
            \item គេឱ្យ $ x:y:z=6:8:15 $ និង $ x-y=98\, g $~។ ចូររកតម្លៃនៃ $ x,y $ និង $ z $~។
            \item គេឱ្យ $ a:b:c=4:6:9 $~។ ចូររកតម្លៃនៃ $ a+b+c $ បើ $ b-a=34\,m $~។
        \end{enumerate}
        \item ចូររកតម្លៃនៃ
        \begin{enumerate}
            \item សៀវភៅសរសេរ $ 12 $ ក្បាល បើសៀវភៅសរសេរ $ 6 $ ក្បាលថ្លៃ $ 4 800\,\textnormal{៛} $ ដោយដឹងថាសៀវភៅសរសេរនីមួយៗមានតម្លៃដូចគ្នា។
            \item តែ $ 10\, kg $ បើតែ $ 3\,kg $ ថ្លៃ $ 45 000\,\textnormal{៛} $~។
        \end{enumerate}
    \end{enumerate}
\end{document}