\documentclass[a4paper,12pt,blue]{pptec}
\begin{document}
	\tableofcontents
	\part{ចំណេះដឹងមូលដ្ឋាន}
	\chapter{ការប្រមូលផ្ដុំទិន្នន័យ (ថ.៨)}
	\section{ទិន្នន័យដែលមិនប្រមូលផ្ដុំជាថ្នាក់}
	\begin{example}
		ការស្រង់ចំនួនកូនក្នុងចំណោម $ 24 $ គ្រួសារ គេបានទិន្នន័យដូចខាងក្រោម៖
		\begin{equation*}
		\begin{array}{cccccccc}
		2 & 0 & 1 & 2 & 4 & 3 & 1 & 2 \\
		2 & 1 & 3 & 4 & 3 & 3 & 2 & 2\\
		3 & 2 & 5 & 3 & 2 & 2 & 4 & 4
		\end{array}
		\end{equation*}
		គេបានតារាងបំណែងចែកប្រកង់
		\begin{table}[H]
			\centering
			\begin{tabular}{|c|l|c|}
				\hline
				ចំនួនកូន $ x $ & របាប់ចំនួនដង & ប្រកង់ $ f $\\
				\hline
				0 & / & 1\\
				\hline
				1 & /// & 3\\
				\hline
				2 & \cancel{////} //// & 9\\
				\hline
				3 & \cancel{////} / & 6\\
				\hline
				4 & //// & 4\\
				\hline
				5 & / & 1\\
				\hline
				ប្រកង់សរុប & & 24\\
				\hline
			\end{tabular}
		\end{table}
	\end{example}
%
\begin{example}
	គ្រូសួរសិស្ស $ 38 $ នាក់អំពីចំណង់ចំណូលចិត្តដែលសិស្សម្នាក់ៗ ចង់រៀនភាសាបរទេសមានអង់គ្លេស $ E $ បារាំង $ F $ រូស៊ី $ R $ ចិន $ C $ ជប៉ុន $ J $ និងកូរ៉េ $ K $ បានម្លើយដូចខាងក្រោម៖
	\begin{equation*}
	\begin{array}{ccccccccccccccccccc}
	E & E & R & F & J & E & E & C & F & E & J & C & F & J & E & C & K & J & E\\
	F & K & E & J & E & C & J & C & E & F & E & E & F & J & C & F & E & K & E
	\end{array}
	\end{equation*}
	\begin{enumerate}[k]
		\item សង់តារាងបំណែងចែកប្រេកង់នៃភាសាបរទេសដែលសិស្សចង់រៀន
		\item តើភាសានីមួយៗដែលសិស្សចង់រៀនមានចំនួនប៉ុន្មាននាក់?
	\end{enumerate}
	\end{example}
	%
	\begin{example}
		ទិន្នន័យចំពោះចំនួនថ្ងៃដែលសិស្ស $ 45 $ នាក់នៅក្នុងថ្នាក់រៀនមួយត្រូវបានអវត្តមានពីសាលាក្នុងមួយឆ្នាំសិក្សាមានដូចខាងក្រោម៖
		\begin{equation*}
		\begin{array}{ccccccccccccccc}
		4 & 3 & 0 & 5 & 2 & 2 & 0 & 1 & 6 & 2 & 7 & 6 & 3 & 3 & 0\\
		4 & 4 & 2 & 0 & 1 & 4 & 0 & 2 & 1 & 5 & 1 & 6 & 3 & 2 & 5\\
		2 & 2 & 4 & 0 & 6 & 3 & 2 & 0 & 1 & 7 & 5 & 0 & 1 & 2 & 3
		\end{array}
		\end{equation*}
		ចូរសង់តារាងបំណែងចែកប្រេកង់។s
	\end{example}
	\section{ទិន្នន័យដែលប្រមូលផ្ដុំជាថ្នាក់}
	\begin{example}
		ខាងក្រោមនេះជាពិន្ទុភាសាអង់គ្លេសនៃសិស្ស $ 40 $ នាក់។
		\begin{equation*}
		\begin{array}{cccccccccc}
		46 & 58 & 65 & 70 & 75 & 48 & 59 & 66 & 71 & 78\\
		51 & 59 & 66 & 72 & 79 & 52 & 60 & 66 & 72 & 80\\
		54 & 62 & 67 & 73 & 82 & 55 & 63 & 68 & 73 & 83\\
		55 & 64 & 68 & 73 & 84 & 56 & 65 & 69 & 74 & 89
		\end{array}
		\end{equation*}
		\begin{enumerate}[k]
			\item បែងចែកទិន្នន័យជា $ 9 $ ថ្នាក់ដែលមានប្រវែងស្មើគ្នាដោយជ្រើសរើសទំហំថ្នាក់សមស្របមួយ។
			\item សង់តារាងបំណែងចែកប្រេកង់។
		\end{enumerate}
	\end{example}
	\begin{answer}
	តារាងបំណែងចែកប្រេកង់
	\begin{table}[H]
		\centering
		\begin{tabular}{|c|c|l|c|}
			\hline
			ថ្នាក់ & ចន្លោះថ្នាក់ & របាប់ចំនួនដង & ប្រេកង់ $ f $\\
			\hline
			1 & 45-<50 & // & 2\\
			\hline
			2 & 50-<55 & /// & 3\\
			\hline
			3 & 55-<60 & \cancel{////} / & 6\\
			\hline
			4 & 60-<65 & //// & 4\\
			\hline
			5 & 65-<70 & \cancel{////} //// & 9\\
			\hline
			6 & 70-<75 & \cancel{////} /// & 8\\
			\hline
			7 & 75-<80 & /// & 3\\
			\hline
			8 & 80-<85 & //// & 4\\
			\hline
			9 & 85-<90 & / & 1\\
			\hline
			& ប្រេកង់សរុប & & 40\\
			\hline
		\end{tabular}
	\end{table}
	\end{answer}
	%
	\begin{example}
		ការអង្កេតអាយុរបស់គ្រូបង្រៀននៅក្នុងសាលាមួយទទួលបានទិន្នន័យដូចខាងក្រោម៖
		\begin{table}[H]
			\centering
			\begin{tabular}{cccccccccc}
				25 & 50 & 51 & 56 & 39 & 42 & 45 & 30 & 49 & 42\\
				46 & 59 & 45 & 43 & 52 & 26 & 53 & 34 & 53 & 64\\
				57 & 46 & 42 & 35 & 28 & 58 & 38 & 46 & 33 & 47\\
				40 & 48 & 61 & 44 & 31 & 39 & 44 & 22 & 55 & 54\\
				32 & 42 & 47 & 37 & 56 & 36 & 41 & 54 & 42 & 54
			\end{tabular}
		\end{table}
		\begin{enumerate}[k]
			\item រៀបចំទិន្នន័យតាមលំដាប់ដោយផ្គុំជា $ 9 $ ថ្នាក់ដាក់ក្នុងតារាងបំណែងចែកប្រេកង់
			\item តើគ្រូបង្រៀនមានអាយុតិចជាង $ 50 $ ឆ្នាំមានចំនួនប៉ុន្មាននាក់?
		\end{enumerate}
	\end{example}
	%
	\begin{answer}
		តារាងបំណែងចែកប្រេកង់
		\begin{table}[H]
			\centering
			\begin{tabular}{|c|c|c|c|}
				\hline
				ថ្នាក់ & ចន្លោះថ្នាក់ & របាប់ចំនួនដង & ប្រេកង់\\
				\hline
				1 & 20-<25 & &\\
				\hline
				2 & 25-<30 & &\\
				\hline
				3 & 30-<35 & &\\
				\hline
				4 & 35-<40 & &\\
				\hline
				5 & 40-<45 & &\\
				\hline
				6 & 45-<50 & &\\
				\hline
				7 & 50-<55 & &\\
				\hline
				8 & 55-<60 & &\\
				\hline
				9 & 60-<65 & &\\
				\hline
				& ប្រេកង់សរុប & & \\
				\hline
			\end{tabular}
		\end{table}
	\end{answer}
	%
	\section{អ៊ីស្តូក្រាម}
	\section{លំហាត់}
	%
	\chapter{ស្ថិតិតាង (ថ.៩)}
	\section{ប្រកង់ ប្រកង់ធៀប}
	\section{ប្រេកង់កើន ប្រកង់ថយ}
	\section{អ៊ីស្តូក្រាមដែលមានអំព្លីទុតស្មើគ្នា}
	\section{អ៊ីស្តូក្រាមដែលមានអំព្លីទុតស្មើគ្នា}
	\section{លំហាត់}
	%
	\chapter{ស្ថិតិវិភាគ (ថ.៩)}
	\section{ម៉ូត}
	\section{មេដ្យាន}
	\section{មធ្យម}
	\section{គណនាម៉ូត មេដ្យាន តាមតារាងប្រកង់}
	\section{លំហាត់}
	%
	\chapter{ប្រូបាប (ថ.៩)}
	\section{ប្រកង់ធៀប}
	\section{ប្រូបាប}
	\section{ផលបូកប្រូបាប}
	\section{ផលគុណប្រូបាប}
	\section{លំហាត់}
	%
	\part{ចំណេះដឹងមធ្យមសិក្សាបឋមភូមិ}
	\chapter{ប្រភាគពីជគណិត (ថ.៨)}
	\section{សញ្ញាណប្រភាគពីជគណិត}
	\section{តម្លៃលេខនៃប្រភាគពីជគណិត}
	\section{សម្រួលប្រភាគពីជគណិត}
	\section{ប្រមាណវិធីលើប្រភាគពីជគណិត}
	\section{លំហាត់}
	%
	\chapter{សមីការដឺក្រេទី១មានមួយអញ្ញាត (ថ.៩)}
	\section{សមីការដឺក្រេទី១មានមួយអញ្ញាត}
	\section{ដោះស្រាយសមីការដឺក្រេទី១មានមួយអញ្ញាត}
	\section{ចំណោទសមីការដឺក្រេទី១មានមួយអញ្ញាត}
	\section{លំហាត់}
	%
	\chapter{ប្រព័ន្ធសមីការដឺក្រេទី១មានពីរអញ្ញាត (ថ.៨)}
	\section{សមីការដឺក្រេទី១មានពីរអញ្ញាត}
	\section{ក្រាភិចនៃសមីការ $ Ax+By=C $}
	\section{ប្រព័ន្ធសមីការដឺក្រេទី១មានពីរអញ្ញាត}
	\section{ចំណោទប្រព័ន្ធសមីការ}
	\section{លំហាត់}
	%
	\chapter{ឬសទី $ n $ (ថ.៩)}
	\section{រំលឹកអំពីស្វ័យគុណ}
	\section{ឬសទី $ n $}
	\section{ការបញ្ចេញបញ្ចូលមួយចំនួនក្នុងរ៉ាឌីកាល់}
	\section{ប្រមាណវិធីលើរ៉ាឌីកាល់}
	\section{វិធីបំបាត់រ៉ាឌីកាល់ពីភាគបែង}
	\section{លំហាត់}
	%
	\chapter{វិសមីការដឺក្រេទី១ មាន១អញ្ញាត (ថ.៩)}
	\section{វិសមភាព}
	\section{វិសមីការដឺក្រេទី១ មាន១អញ្ញាត}
	\section{ប្រព័ន្ធវិសមីការដឺក្រេទី១ មាន១អញ្ញាត}
	\section{ដោះស្រាយចំណោទវិសមភាព វិសមីការ និងប្រព័ន្ធវិសមីការ}
	\section{លំហាត់}
	%
	\chapter{ត្រីកោណ (ថ.៧)}
	\section{និយមន័យ}
	\section{ធាតុដទៃទៀតនៃត្រីកោណ}
	\section{ត្រីកោណពិសេស}
	\section{មុំបាតនៃត្រីកោណសមបាទ}
	\section{ផលបូកមុំក្នុងត្រីកោណ}
	\section{មុំក្រៅត្រីកោណ}
	\section{សំណង់ត្រីកោណ}
	\section{លំហាត់}
	%
	\chapter{ករណីប៉ុនគ្នានៃត្រីកោណ (ថ.៨)}
	\section{និយមន័យ}
	\section{ករណីប៉ុនគ្នានៃត្រីកោណសាមញ្ញ}
	\section{ករណីប៉ុនគ្នានៃត្រីកោណកែង}
	\section{លំហាត់}
	%
	\chapter{វិសមភាពនៃត្រីកោណ (ថ.៨)}
	\section{ទំនាក់ទំនងរវាងមុំ និងជ្រុង}
	\section{ទំនាក់ទំនងរវាងជ្រុងនៃត្រីកោណមួយ}
	\section{លំហាត់}
	%
	\chapter{បន្ទាត់ និងអង្កត់ ជួបក្នុងត្រីកោណមួយ (ថ.៨)}
	\section{លក្ខណៈកន្លះបន្ទាត់ពុះមុំនៃត្រីកោណមួយ}
	\section{លក្ខណៈមេដ្យាទ័រនៃត្រីកោណ}
	\section{លក្ខណៈមេដ្យាននៃត្រីកោណ}
	\section{លក្ខណៈកម្ពស់នៃត្រីកោណ}
	\section{លំហាត់}
	%
	\chapter{ចតុកោណ (ថ.៨)}
	\section{លក្ខណៈទូទៅនៃចតុកោណប៉ោង}
	\section{ចតុកោណពិសេស}
	\section{អនុវត្តន៍}
	\section{លំហាត់}
	%
	\chapter{ប្រលេឡូក្រាម (ថ.៨)}
	\section{និយមន័យ}
	\section{លក្ខណៈនៃប្រលេឡូក្រាម}
	\section{លក្ខណៈសម្គាល់ប្រលេឡូក្រាម}
	\section{អនុវត្តន៍ចំពោះត្រីកោណ}
	\section{លំហាត់}
	%
	\chapter{ចតុកោណព្នាយ (ថ.៨)}
	\section{និយមន័យ}
	\section{លក្ខណៈចតុកោណព្នាយ}
	\section{លក្ខណៈចតុកោណព្នាយសមបាត}
	\section{លំហាត់}
	%
	\chapter{ប្រលេឡូក្រាមពិសេស (ថ.៨)}
	\section{និយមន័យ}
	\section{លក្ខណៈចតុកោណកែង ចតុកោណស្មើ ការេ}
	\section{អនុវត្តន៍ក្នុងត្រីកោណកែង}
	\section{លំហាត់}
	%
	\part{វិធីសាស្ត្របង្រៀន}
	\chapter{ទ្រឹស្ដី}
	\section{សមត្ថភាពមូលដ្ឋាននៃគណិតវិទ្យាសម្រាប់សិស្ស}
	\section*{មធ្យមសិក្សាបឋមភូមិ (ToP ថ្នាក់ទី ៧,៨,៩)}
	\section{ការណែនាំប្រើប្រាស់សៀវភៅសិស្ស និងសៀវភៅគ្រូ (ToP ថ្នាក់ទី ៧,៨,៩)}
	\section{គោលវិធីសិស្សមជ្ឈមណ្ឌល}
	\section{កិច្ចតែងការបង្រៀន (ToP ថ្នាក់ទី ១១)}
	\section{របៀបបង្រៀននិយមន័យ និងទ្រឹស្ដីបទ}
	%
	\chapter{អនុវត្តន៍ និងហ្វឹកហ្វឺនគរុកោសល្យ}
	\section{អនុវត្តន៍ការសរសេរកិច្ចតែងការបង្រៀន}
	\section{ការឡើងបង្រៀនសាកល្បង}
	\section{ការធ្វើថ្នាក់និទស្សន៍}
	%
\end{document}